\documentclass[a4paper,12pt]{article}
\usepackage[left=2.0cm, right=2.0cm, top=2.5cm, bottom=2.0cm]{geometry}
\usepackage{indentfirst}
\usepackage{amsmath, amssymb, amsfonts}
\usepackage{multirow}
\usepackage{graphicx}
\usepackage{xeCJK}
\usepackage{fancyhdr}
\usepackage{verbatim}
\usepackage{algorithm}
\usepackage{algpseudocode}
\usepackage{hyperref}
\pagestyle{fancy}
\lhead{}
\chead{图形学大作业系统报告}
\rhead{陈劭源(161240004)}
\lfoot{}
\cfoot{}
\rfoot{\thepage}
\setCJKmainfont[BoldFont=SimHei,ItalicFont=KaiTi]{SimSun}
\setCJKmonofont{KaiTi}
\setmainfont{Times New Roman}
\setmonofont{Courier New}
\renewcommand\refname{参考文献}
\graphicspath{{figures/}}

\title{图形学大作业系统报告}
\author{陈劭源(161240004) \\ \href{mailto:sy\_chen@smail.nju.edu.cn}{sy\_chen@smail.nju.edu.cn}}
\date{\today}

\begin{document}
\maketitle

\section{综述}



\section{算法介绍}
绘制曲线$f(x,y) = 0$的基本原则是:当$\left|\frac{\mathrm{d}y}{\mathrm{d}x} \big|_{(x_0, y_0)} \right| \leq 1$时,沿$x$轴递进采样画点;当$\left|\frac{\mathrm{d}y}{\mathrm{d}x} \big|_{(x_0, y_0)} \right| > 1$时,沿$y$轴递进采样画点。这样可以保证相邻两个绘制点$(x_i, y_i), (x_{i+1}, y_{i+1})$之间满足$\max\{|x_i - x_{i+1}|, |y_i - y_{i+1}|\} = 1$。

\subsection{DDA算法}

DDA算法是利用对曲线微分方程积分的方法来绘制曲线的。DDA算法通常用于绘制线段、多边形等,但也可用来绘制非线性曲线\cite{wiki:DDA}。利用DDA算法绘制线段的伪代码如下:

\begin{algorithm}[htb] 
\caption{DDA画线算法} 
\label{alg:DDA} 
\begin{algorithmic}[1] 
\Require 
线段的两个端点$(x_1, y_1)$, $(x_2, y_2)$。假定$x_1 < x_2, |x_2 - x_1| \geq |y_2 - y_1|$。
\State $y = y_1, k = \frac{y_2 - y_1}{x_2 - x_1}$
\For{$x = x_1$ to $x_2$}
    \State plot $[x], [y]$
    \State $y = y + k$
\EndFor
\end{algorithmic} 
\end{algorithm}

\subsection{\dots}
\dots
		
\section{系统介绍}
\begin{figure}
\centering
\includegraphics[width=\linewidth]{uml.pdf}
\caption{系统的UML类图}
\end{figure}
\section{总结}
\dots

\bibliographystyle{plain}%
%"xxx" should be your citing file's name.
\bibliography{report}
\end{document}
